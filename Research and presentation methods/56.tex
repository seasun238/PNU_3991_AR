{\documentclass [10pt,a4paper,]{book}
	
\usepackage{amsmath}
	
\begin{document}
		
\begin{flushright}
			
 \textbf{CHAPTER FIVE}      
			
\end{flushright}
...............................................................................................................
\begin{flushright}
ETHICS AND THE e-RESEARCHER
\end{flushright}
\begin{flushright} 
Penetrating so many secrets, we cease to believe in the unknowable. But there 

it sits nevertheka, calmly licking its chops. 


I-1. L. Mencken. Arnencan editor and critic 

\end{flushright}
Researchers associated with academic institutions need to submit an ethics proposal prior to conducting their research. Researchers who are associated with private or commercial research functions will also need to think and plan vet). seriously about the ethical implications of their c-research. Hence, it is fitting to begin a discussion here of ethical procedures for e-research. As an e-researcher, you will encounter a number of dilemmas, issues, and problems with respect to ethics. These issues pertain to all types of research, but they have a tendency to acquire added and more complex twists when undertaken in an electronic format. Irrespective of the size, complexity, or methodology employed in your research, you must always adhere to ethical and moral principles. We believe that when all is said and done ethical research is really more about creating and maintaining respectful relationships with the participants of the study than about formalized codes of ethics and internal review board rules and guide-lines. Unfortunately, many of the existing principl. for face-to-face-based research do not adequately account for significant differences encountered when conducting research on the Net, nor do many of the ethics review committee members understand how c-research is actually conducted and, hence, may not recognize an ethically com-promised Net-based study. As such it is up to us, as competent e-researchers and prin-cipled human beings, to define and practice ethical behavior. This requires personal integrity and self-regulation, which includes openness and honesty about all aspects of the study, as well as reflection on our actions before, during, and after conducting a research project.


This chapter is not about following codes of ethics ter preparing for institutional ethics review boards. All academic institutions have prescriptive guidelines defining what constitutes ethical behavior to which all researchers associated with these institutions 
\begin{flushleft}
56
\end{flushleft}
\end{document}
