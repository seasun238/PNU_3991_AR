\documentclass [12pt]{beamer}
\usepackage{xcolor}	
\usepackage{tikz}
\usetheme{Warsaw}
\useoutertheme{infolines}
\usepackage{ragged2e}
\begin{document}

\section*{kholase safahat 55...57}
\subsection*{helen dabaghi  }	
\begin{frame}
\justifying	
Researchers involved in private or commercial research activities should also think and plan very seriously about the ethical implications of their e-research.  It is therefore worthwhile to begin the discussion here on the ethical methods of electronic research.
\end{frame}

\begin{frame}
\justifying	
As electronic researchers and competent human beings, we must define and practice ethical behaviors.  This requires personal integrity and self-regulation, which includes openness and honesty about all aspects of the study, as well as reflection on our actions before, during and after a research project.
\end{frame}

\begin{frame}
\justifying	
There are some hard and fast rules that definitively define ethical research, yet there are some agreed upon principles.  These principles include: (1) voluntary informed consent;  (2) privacy, confidentiality and anonymity;  And (3) identify the elements of research risk.
\end{frame}
\end{document}
