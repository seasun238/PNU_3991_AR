{\documentclass [10pt,a4paper,]{book}
	
\usepackage{amsmath}
	
\begin{document}
		
\begin{flushright}
			
ETHICS AND THE E-RESE.CHER       \textbf{57 }
			
\end{flushright}
must adhere. Nor is this chapter an attempt to provide a definitive moral platform to guide research conducted on the Net. Rather, the chapter is designed to bring to your attention the ethical complexities of e-research that must be understood and inte-grated into the practice of an e-researcher. 


Before we begin a discussion on how the Net complicates ethics, it is important to acknowledge that there are few hard-and-fast rules that categorically define ethical research, however there are a few agreed-upon principles. These principles include: (1) voluntary informed consent; (2) privacy, confidentiality, and anonymity; and (3) recognizing the elements of research risk (Bickman  Rog, 1998). In practice these principles can conflict with each other, and the researchers will then need to balance carefully the importance of advancement of understandings and knowledge and the need to guard against potential harm to the research participants (e.g., loss of dignity, self-esteem, privacy).
\begin{flushleft} 
HOW THE NET COMPLICATES ETHICS
\end{flushleft} 
Ethical concerns are becoming increasingly multifarious in our postmodern society, which is defined by complexity, multiculturalism, and media saturation. Researchers are expected to understand the basic principles of good research and be cognizant with respect to any proscriptive codes of ethics. Ultimately, however, ethical conduct depends on the individual researcher. "The researcher has a moral and professional obligation to be ethical, even when research subjects are unaware of or unconcerned about ethics" (Neuman, 2000). 


In any new field of study, time and experience are required before it is possible to infer the appropriate definition and extent of ethical behavior. Infamous studies of potentially harmful deception (Milgram, 1963) in social psychology illustrate the need for ethical behavior by researchers and the usefulness of institutionalized guidelines to both define and enforce ethical practice by researchers. 


The study that perhaps best brings to our attention the immediate need to estab-lish ethical protocols for Net-based research is the Rimm study reported by Thomas (1996). This long-term research study involved an analysis of data about users of text-based erotica files that was collected via analysis of particular Usenet posts from the alt.binaries hierarchy. The research study also included a summary of the statistics on Usenet readership. Readership data was obtained from the private files of users on a large university computer system. The study was described by the researcher (Marty Rimm, who was at that time an undergraduate student) to be scientific. comprehensive, and the first of its kind. The study involved three main data-gathering techniques. The initial pool of subjects was identified from approximately 1,000 Bulletin Board Systems (BBS). From this pool, ninety-one were chosen and thirty-five were used. Descriptions of pornographic files were then downloaded for analysis by a linguistic parsing soft-ware script. At least half of the BBSs were not accessible to the general public (e.g., proof of age was a requirement for access). At the researcher's request, the system operator collected other data including demographic details (age, sex, nationality, marital status, position, department, and other confidential information) of the users
\end{document} 
